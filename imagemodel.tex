%%% FIXME:
%%%---I think we assume 100% fill factor also
%%%---and well-sampled images?

%%% HOGG SAY: we assume *either* well sampled *or* complete fill factor.

% To begin, let us assume we have an image containing a point source
% that we wish to detect.  Assume that the image contains additive
% Gaussian noise that is independent (uncorrelated) between pixels and
% equal in variance.  This is roughly equivalent to assuming that the
% noise in the image is dominated by sky noise or read noise, rather
% than Poisson noise from the source itself.  This is close to correct
% at the limit of detecting faint sources in wide-band optical images.
% Also assume that we have a good model of the point-spread function,
% and that it is constant over the image.  Further, let us assume that
% the image has had the sky and other ``background'' levels removed so
% that the mean value would be zero if the image did not contain any
% sources.  

We consider astronomical images such as those obtained from an
(idealized) CCD.  Aside from noise, each pixel contains a signal
related to the number of photons incident upon it during the exposure.
%
We are measuring the intensity field $I(\thetavec,\posvec,\lambda,t)$
(energy per unit solid angle, two-dimensional transverse position,
wavelength, and time) by taking an exposure for some amount of time
through a filter, optics, and possibly atmosphere.  This can be seen
as an integration over time, wavelength, telescope aperture, and the
point-spread function.
%
% We model the signal $C_{\jvec}$ as being related to the intensity
% field $I(\thetavec,\posvec,\lambda,t)$ (energy per unit solid angle,
% two-dimensional transverse position, wavelength, and time) by a
% six-dimensional integral
Consider a discrete image made up of a square array of pixels $\jvec$,
each of which has signal (or counts) $S_{\jvec}$, where we use $\jvec$
as a two-dimensional focal-plane position, measured in pixel units.
The signal $S_{\jvec}$ is given by the integral
\begin{eqnarray}\displaystyle
%C_{\jvec} &=& \frac{\dd C}{\dd E}\,\int\dd^2\thetavec'\,\psi(\thetavec_{\jvec} - \thetavec')\,\int \dd^2\posvec\,A(\posvec)\int \dd\lambda\,\lambda\,R(\lambda)\,\int\dd t\,H(t)\,I(\thetavec',\posvec,\lambda,t) + \noise_{\jvec}
%\\
%C_{\jvec} &=& \frac{\dd C}{\dd E}\,\int \dd^2\cvec\,\psi(\jvec - \cvec)\,\Omega_1\,\int \dd^2\posvec\,A(\posvec)\,\int \dd\lambda\,\lambda\,R(\lambda)\,\int\dd t\,H(t)\,I(\thetavec_{\cvec},\posvec,\lambda,t) + \noise_{\jvec}
S_{\jvec} &=& \frac{\dd C}{\dd E}\,\Omega_1\, %
\!
\int \!\!\!
\int \!\!\!
\int \!\!\!
\int
I(\thetavec_{\cvec},\posvec,\lambda,t)
\;H(t)\,
\dd t\,
\;
\lambda\,R(\lambda)\,
\dd\lambda\,
\;
A(\posvec)\,
\dd^2\posvec\,
\;
\psi(\jvec - \cvec)\,
\dd^2\cvec\,
 + \noise_{\jvec}
\quad,
\label{eqn:bigint}
\end{eqnarray}
where 
%
$\dd C/\dd E$ is the calibration factor taking integrated energies to per-pixel counts (in the mean),
%
$\Omega_1$ is the solid angle of a single pixel (assumed the same for all pixels),
%
$I(\cdot)$ is the intensity field,
%
$\thetavec$ is a two-dimensional angular position on the sky,
%
$t$ is a time, $H(t)$ is a function describing the time period over which the telescope was
integrating this exposure,
%
$\lambda$ is a wavelength, $R(\lambda)$ is a function describing the bandpass \citep{kcorrect},
%
$\posvec$ is a two-dimensional position vector in the telescope aperture plane,
%
$A(\posvec)$ is a function describing the telescope aperture,
%
$\psi(\cdot)$ is a function describing the pixel-convolved point-spread function of the
whole system,
%
$\cvec$ is a two-dimensional focal-plane position in the same coordinates and units as $\jvec$,
%
and
$\noise_{\jvec}$ is the noise contribution to pixel $\jvec$.
%
The per-pixel noise could include effects such as read noise and dark current.
%
Importantly, the PSF function is the pixel-convolved PSF, not simply
the PSF due to optics and atmosphere, so image pixels are a sampling
and not a convolution of the PSF-convolved intensity field.
%We will assume the PSF is well-sampled by the pixels.

In %what follows
the rest of this \doctype, we will assume that the (perhaps tiny) image of
interest contains only a uniform background value plus one point
source $k$ with some (unknown) flux $F_k$ and position $\thetavec_k$.
In this case, the intensity field is just
\begin{eqnarray}\displaystyle
I(\thetavec',\posvec,\lambda,t) &=& I_{\sky}(\lambda, t) + F_k(\lambda, t)\,\delta(\thetavec_k - \thetavec')
\quad,
\label{eqn:intensity}
\end{eqnarray}
where $I_{\sky}$ is the intensity of the background (sky), $F_k$ has
units of flux (energy per area per time per wavelength), and
$\delta()$ is a two-dimensional delta function on the sky.  For
simplicity, we have dropped all assumption-violating contributions
from galaxies, nebulosity, neighboring sources, or sky gradients.  The
noise values are drawn from a probability distribution which we assume
to be
\begin{eqnarray}\displaystyle
p(\noise_{\jvec}) &=& \gaussian{\noise_{\jvec}|0,\sigma_1^2}
\quad,
\end{eqnarray}
where $\gaussian{x|m,V}$ is the Gaussian distribution with mean $m$
and variance $V$, and $\sigma_1^2$ is the per-pixel noise variance
(assumed equal for all pixels).  That is, we are assuming constant,
zero-mean, Gaussian noise.  This assumption is not too bad for
sky-limited images from which the sky level has been subtracted: the
sky contribution can be modeled as a Poisson process contributing
Poisson or ``shot'' noise which we then approximate by the Gaussian.
This approximation gets better as the sky level (number of
photo-electrons) increases.

In the above, we have made extremely strong assumptions.  In addition
to the listed assumptions we have assumed that any electronic gain at
readout has been calibrated out.  We have also assumed that the image
is not contaminated by cosmic rays, stray light, bad pixels, bad
columns, electronic artifacts, or any other of the many defects in
real images.


\subsubsection{Single point source in a single image}

In this section, we will assume that the (perhaps tiny) image of
interest contains only a single isolated point source $k$ with some
constant flux
$F_k(\lambda)$, and constant sky intensity $I_{\sky}(\lambda)$.
Applying the integral of \eqnref{eqn:bigint}\ to the simple intensity
model (\eqnref{eqn:intensity}) with a constant-flux source, the
aperture $A(x)$ and exposure $H(t)$
% and sensitivity $R(\lambda)$
terms become constants which we will combine with the $\Omega_1$ and
$\frac{\dd C}{\dd E}$ terms, leaving
\begin{eqnarray}\displaystyle
%   C_{\jvec} &=& K \int \dd^2 \cvec\, \psi(\jvec - \cvec)
%   \int \dd\lambda\,\lambda\,R(\lambda)\,
%   \left(
%   I_{\sky}(\lambda) + F_k(\lambda)\,\delta(\thetavec_k - \thetavec_{\cvec})
%   \right)  + \noise_{\jvec} \quad ,
S_{\jvec} &=& K
\int \!\!
\int
\Big[
  I_{\sky}(\lambda) + F_k(\lambda)\,\delta(\thetavec_k - \thetavec_{\cvec})
\Big]
\lambda\,R(\lambda)\,
\dd\lambda\,
\psi(\jvec - \cvec)\,
\dd^2 \cvec\
  + \noise_{\jvec} \quad ,
\end{eqnarray}
and integrating out the sky and source terms yields
\begin{eqnarray}\displaystyle
  S_{\jvec} &=& C_{\sky} + C_k \, \psi(\jvec - \kvec)  + \noise_{\jvec} \quad ,
\end{eqnarray}
%where $C_k$ is a pixel-summed count, and $\kvec$ is the pixel position
%corresponding to the angular direction $\thetavec_k$ of source $k$.
where $\kvec$ is the pixel position corresponding to the angular
direction $\thetavec_k$ of source $k$, and $C_k$ is the total number
of counts (summed over pixels) from the source, which we will call the
\emph{source counts}.


We will subtract the sky level $C_{\sky}$, leaving the sky-subtracted
image $C_{\jvec}$.  We assume that we can estimate the sky level
exactly; in reality our estimate will be noisy.  The sky term
$C_{\sky}$ contributes Poisson noise to the image, which we will
approximate by Gaussian noise and combine with the other noise term
$\noise_{\jvec}$, yielding
\begin{eqnarray}\displaystyle
  C_{\jvec} &\drawnfrom& C_k \, \psi(\jvec - \kvec) + \gaussx{0,\sigma_1^2}
\label{eq:modelimg}
\end{eqnarray}
where the symbol ``$\drawnfrom$'' means ``drawn from the
distribution'', and $\gaussx{\mu,\sigma^2}$ is the Gaussian
distribution with mean $\mu$ and variance $\sigma^2$.  Put simply, in
our model the image pixels contain signal due to a pixel-convolved
point-spread function $\psi(\cdot)$ shifted to the source center
$\kvec$ and scaled by the source counts $C_k$; plus additive,
pixelwise-independent, constant-variance Gaussian noise.  We will
assume that the point-spread function $\psi(\cdot)$ is non-negative
and, when evaluated on a pixel grid, sums to unity.
% don't need a fig for this
%See \figref{fig:image}.

% The signal-to-noise ratio in a pixel $S_{\jvec}$ is the model mean signal
% over the per-pixel noise standard deviation:
% \begin{equation}
% \snr{S_{\jvec}} = \frac{C_k \, \psi(\jvec - \kvec)}{\sigma_1} \quad .
% \end{equation}
